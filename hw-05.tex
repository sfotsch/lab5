ss{article}
\usepackage{verbatim}
\usepackage[margin=1in]{geometry}
\usepackage{amsmath}
\usepackage[english]{babel}
\usepackage{listings}
\lstset{language=C,basicstyle=\ttfamily,breaklines=true}
\usepackage[autostyle=true, english = american]{csquotes}
\MakeOuterQuote{+}

\newcommand{\n}{\par\vspace{\baselineskip}}

\title{Homework 5}
\author{Shawn Fotsch}
\date{\today}

\begin{document}
   \maketitle

   \section{Exercise 2}
      \begin{itemize}
      \item Print the names of players that throw with the left arm. \n

      \item Print the whole record for players that throw and bat with different arms. \n

      \item Print the table without header lines (i.e., only playerâ<80><99>s records). \n

      \item Print the names of players that were born between 1930 and 1940 (inclusive). \n

      \item Print the whole table using \verb:;: as field separator, and \verb;\n\n; as record separator. \n

      \item Print the name of the teamâ<80><99>s catchers, without any headers. \n

      \end{itemize}

                \verb;numbers.txt; contains two numbers that are comma separated $a, b$.
   \section{Exercise 3}
      \begin{itemize}
      \item Write an \verb;awk; command line to output the largest of the two for each line of the file.\n

      \item Write an \verb;awk; command line to compute the average per column. (Do not hard-code the number of rows.)\n

      \end{itemize}

   \section {Exercise 4}
      Write an \verb;awk; script named \verb;psroot.awk; to parse the output of the \verb;ps au; command and give the total \%CPU of the user \verb;root;. \n

                        %\lstinputlisting{psroot.awk}

   \section {Exercise 5}
                Recall that the roots of the equation $ax^{2} +bx+c = 0$ can be solved with the quadratic formula.
                \centering$\frac{-b\pm\sqrt{b^{2}-4ac}}{2a}$.

                \begin{itemize}
                        \item Write an \verb;awk; function to solve the quadratic formula. The function should receive four arguments: three coefficients, and the sign of the square root to compute.\n

                        %\lstinputlisting{quadfunc.awk}
                        \item  \verb;coefs.txt; is a file where each line corresponds to a quadratic equation and the coefficients are listed as a:b:c. Write an \verb;awk; script named \verb;quad.awk; to output the roots of each line (in order) in the form $a : b : c : x_{1} : x_{2}$.  (The function to compute square roots is \verb;sqrt;.)\n

                        %\lstinputlisting{quad.awk}

                \end{itemize}

\end{document}

